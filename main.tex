% !TEX program = xelatex


\documentclass[a4paper,fontsize=14bp]{article}

%%% Работа с русским языком
%\usepackage{cmap}					% поиск в PDF
\usepackage{mathtext} 				% русские буквы в фомулах
%\usepackage[T2A]{fontenc}			% кодировка
\usepackage[utf8]{inputenc}			% кодировка исходного текста
\usepackage[english,russian]{babel}	% локализация и переносы

%%% Дополнительная работа с математикой
\usepackage{amsmath,amsfonts,amssymb,amsthm,mathtools} % AMS
\usepackage{icomma} % "Умная" запятая: $0,2$ --- число, $0, 2$ --- перечисление

%% Номера формул
%\mathtoolsset{showonlyrefs=true} % Показывать номера только у тех формул, на которые есть \eqref{} в тексте.
%\usepackage{leqno} % Немуреация формул слева

%% Свои команды
\DeclareMathOperator{\sgn}{\mathop{sgn}}

%% Перенос знаков в формулах (по Львовскому)
\newcommand*{\hm}[1]{#1\nobreak\discretionary{}
{\hbox{$\mathsurround=0pt #1$}}{}}

%%% Работа с картинками
\usepackage{graphicx}  % Для вставки рисунков
\graphicspath{images/} % папки с картинками
\setlength\fboxsep{3pt} % Отступ рамки \fbox{} от рисунка
\setlength\fboxrule{1pt} % Толщина линий рамки \fbox{}
\usepackage{wrapfig} % Обтекание рисунков текстом

%%% Работа с таблицами
\usepackage{array,tabularx,tabulary,booktabs} % Дополнительная работа с таблицами
\usepackage{longtable}  % Длинные таблицы
\usepackage{multirow} % Слияние строк в таблице
\usepackage{tabto}

\usepackage{cite} % Работа с библиографией
%\usepackage[superscript]{cite} % Ссылки в верхних индексах
%\usepackage[nocompress]{cite} % 
\usepackage{csquotes} % Еще инструменты для ссылок

%%% Заголовок
\author{Ахтямов Д.С.}
\date{\today}

%%% Общие настройки
\usepackage[left=20mm, top=15mm, right=20mm, bottom=15mm, nohead, footskip=10mm]{geometry} % настройки полей документа
\usepackage{setspace}
\onehalfspacing     %% полуторный интервал
\usepackage{fontspec}
\setlength{\parindent}{1.25cm}  %% отступ красной строки
\usepackage{indentfirst}
\setlength{\parskip}{0cm}
\setmainfont[Mapping=tex-text]{Times New Roman}
\usepackage{tocloft}    %% точки в оглавлении
\renewcommand{\cftsecleader}{\cftdotfill{\cftdotsep}}
\usepackage{scrextend} %% для 14 размера шрифта 
\addto\captionsrussian{ %% содержание -> оглавление
  \renewcommand{\contentsname}%
    {Оглавление}%
}

\usepackage{titlesec}
\titleformat{\section}
  {\normalfont\fontsize{14bp}{16.8}\bfseries}{\thesection.}{1em}{}
\titleformat{\subsection}
  {\normalfont\fontsize{14bp}{16.8}\bfseries}{\thesubsection.}{1em}{}
\titleformat{\subsubsection}
  {\normalfont\fontsize{14bp}{16.8}\bfseries}{\thesubsubsection.}{1em}{}
 


%%% Настройка колонтитулов
\usepackage{fancyhdr}
\pagestyle{fancy}
\fancyhf{}
\rfoot{\thepage}
\renewcommand{\headrulewidth}{0pt}
\renewcommand{\footrulewidth}{0pt}
\fancypagestyle{plain}{\pagestyle{fancy}}

\begin{document} % конец преамбулы, начало документа

{\fontsize{12bp}{16bp} \selectfont \begin{titlepage}
    \newpage
    \begin{center}
    МИНИСТЕРСТВО ТРАНСПОРТА РОССИЙСКОЙ ФЕДЕРАЦИИ\\
    \vspace{4ex}
    ФЕДЕРАЛЬНОЕ ГОСУДАРСТВЕННОЕ АВТОНОМНОЕ ОБРАЗОВАТЕЛЬНОЕ\\ 
    УЧРЕЖДЕНИЕ ВЫСШЕГО ОБРАЗОВАНИЯ\\
    «РОССИЙСКИЙ УНИВЕРСИТЕТ ТРАНСПОРТА»(МИИТ)\\
    \vspace{7ex}
    
    Институт управления и цифровых технологий\\
    \vspace{4ex}
    
    Кафедра «Автоматизированные системы управления»\\

    \vspace{12ex}
    
    \fontsize{16bp}{20bp}\selectfont{
        \textbf{ ОТЧЕТ ПО  \\
        ПРОИЗВОДСТВЕННОЙ ПРАКТИКЕ \\
        (ТЕХНОЛОГИЧЕСКОЙ ПРАКТИКЕ) \\
        }
    }
    \vspace{7ex}
     
    \fontsize{12bp}{14bp}\selectfont
    Тема практики: «Разработка программного комплекса решения задач линейного программирования 
    большой размерности на суперкомпьютере»
    \end{center}

    \vspace{7ex}
    
    \begin{flushright}
        \begin{tabular}{lr@{}} %% убрать отступы справа в таблице
            \fontsize{12bp}{18bp}\selectfont
            Выполнил студент группы УВА-413    &  Д.С. Ахтямов\\
            Принял зав. кафедрой АСУ профессор &  Э.К. Лецкий\\    
        \end{tabular}

        \vspace{2.5ex}
        xx мая 2022 года
    \end{flushright}
    
    \vfill
    
    \begin{center} Москва 2020 \end{center}
    \thispagestyle{empty} % выключаем отображение номера для этой страницы
     
\end{titlepage}}  % титульный лист


% содержание
\newpage 
{\tableofcontents} 

% введение
\newpage
\section*{\centering Введение}
\addcontentsline{toc}{section}{Введение}

В эпоху цифровизации объемы передаваемых и обрабатываемых данных растут стремительными темпами, 
что особенно прослеживается в информационных системах связанными с транспортной отраслью.
Одной из важнейших задач при этом является линейная оптимизация, которая заключается в поиске
оптимального решения на множествах n-мерного пространства, задаваемого системами линейных уравнений
и неравенств, при этом размерности задач могут принимать экстремальные значения. Большое число 
переменных накладывает сильные ограничения на память и вычислительную мощность компьютера, что 
не позволяет решать подобные задачи на персональных компьютерах, ввиду их сравнительно низкой 
эффективности. \par
Для решения подобных ресурсоемких вычислительных задач МИИТ приобрел суперкомпьютер
<<МИИТ Т-4700>>, который на момент закупки стал самым мощным в России суперкомпьютером на базе ЦП 
AMD Opteron\textsuperscript{\tiny TM} 2356 (Barcelona). \par
В данной работе рассматривается разработка программного комплекса для решения задач 
линейного программирования большой размерности на данном суперкомпьютере.

%%%
%%%
%%%

%% первая глава
\newpage

\section{Цели и задачи разработки}
Целью практической работы является разработка вычислительного комплекса для решения задач линейного
программирования с учетом экстремально больших размерностей, а так же 
специфики программного и аппаратного обеспечения суперкомпьютера <<МИИТ Т-4700>>.
Для этого необходимо решить следующие задачи:
\begin{itemize}
  \item Изучить математическую постановку задачи ЛП;
  \item Изучить основные подходы для решения задач ЛП;
  \item Изучить методы и средства для реализации параллельных алгоритмов;
  \item Изучить особенности программного и аппаратного обеспечения суперкомпьютера;
  \item Реализовать программный комплекс;
  \item Провести замеры скорости алгоритма, сделать выводы.
\end{itemize}


\section{Описание процесса разработки программного комплекса}
  \subsection{Выбор средств разработки}

    \subsubsection{Язык программирования} 
      Для написания приложения был выбран язык программирования C++ стандарта 0х, 
      ввиду следующих факторов:  
      \begin{itemize}
        \item Высокое быстродействие;
        \item Наличие мощных инструментов для работы со структурами данных;
        \item Совместимость с установленным на кластере пакетом OpenMPI 1.8.1 \cite{cluster:varf};
        \item Наличие удобного инструмента для рефакторинга, благодаря которому можно безопасно
        переименовывать символы, менять сигнатуры функций и перемещать элементы по иерархии.
      \end{itemize}

    \subsubsection{Компилятор}  
      На суперкомпьютере <<МИИТ Т-4700>> установлен компилятор GCC 4.4.7, входящий в пакет 
      openmpi поэтому при отладке и тестировании на ПК использовался аналогичный компилятор 
      для обеспечения обратной совместимости. 


    \subsubsection{Среда разработки} 
      В качестве среды разработки была выбрана CLion от компании JetBrains, т.к. 
      Данная IDE обладает обладает рядом преимуществ:
      \begin{itemize}
        \item Нативная поддержка языков программирования С и С++, в том числе современных стандартов;
        \item Позволяет мгновенно перейти к объявлению символа или его использованиям в коде,
        искать классы, файлы и любые другие элементы внутри проекта, а так же легко перемещаться по
        структуре и иерархиям кода;
        \item Позволяет генерировать наиболее часто используемые конструкции автоматически;
        \item Наличие удобного инструмента для рефакторинга, благодаря которому можно безопасно
        переименовывать символы, менять сигнатуры функций и перемещать элементы по иерархии.
      \end{itemize}
      Стоит отметить, что CLion является коммерческим продуктом с платной подпиской, а для 
      разработки описываемой в этой работе программы использовалась студенческая лицензия.

    
    


    \subsubsection{Cистема сборки программного обеспечения} e



\section{Структура разрабатываемого программного комплекса}



\section{Документация по классам и методам программного кода}
у


\section{Инструкция пользователя}
у


\section*{\centering Заключение}
\addcontentsline{toc}{section}{Заключение}

\newpage
\renewcommand{\refname}{Список источников}
%\section*{\centering Список литературы}

\begin{thebibliography}{3}
  \addcontentsline{toc}{section}{\refname}
  \bibitem{cluster:varf} Варфоломеев В.А., Гуцу О.Л. Вычислительные кластеры: 
  Методические указания к лабораторным работам по дисциплине 
  "Высокопроизводительные вычислительные системы железнодорожного транспорта". — М.: МИИТ, 2018. — 36 с.
  \bibitem{lin:landnon} Luenberger D. G. et al. Linear and nonlinear programming. – Reading, MA : Addison-wesley, 1984. – Т. 2.
\end{thebibliography}



\end{document}